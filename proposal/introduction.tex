\section{Introduction}
\paragraph{}
Our project will be based on the paper \cite{Friesen}, in which a mixed convex-combinatorial approach for training hard-threshold networks is proposed. A hard-threshold activation function $f(x)$ takes value $1$ for $x > 0$ and $-1$ otherwise. Such a function has derivative 0 almost everywhere and, as such, cannot be trained using the usual method of gradient descent with backpropagation used for soft-threshold networks.

The authors show how training a hard-threshold network reduces to finding a set of targets (+1,-1) for each unit of the network, which needs to be ``feasible'', that is, needs to satisfy a set of inequalities based on the dataset one wishes to classify. And for a set of feasible targets, a set of weights minimizing a loss function should be found.

Thus, this problem of finding a feasible set of targets and minimizing set of weights is referred as a Mixed Convex-Combinatorial problem. Finding a feasible set of targets would be a combinatorial one, and finding a set of weights minimizing the loss function would be a convex one.

In Section \ref{tools} we will discuss the datasets needed for our project and the programming tools we plan to use. Sections \ref{heuristics}, \ref{adversary}, and \ref{generalization} will describe the three questions we propose to study for this research project.